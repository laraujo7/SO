\documentclass[a4paper]{report}
\usepackage[utf8]{inputenc}
\usepackage[portuguese]{babel}
\usepackage{hyperref}
\usepackage{a4wide}

\hypersetup{pdftitle={Controlo e Monitorização de Processos e Comunicação},
pdfauthor={Carlos Ferreira, José Alves, Luis Araújo},
colorlinks=true,
urlcolor=blue,
linkcolor=black}

\usepackage{subcaption}
\usepackage[cache=false]{minted}
\usepackage{listings}
\usepackage{booktabs}
\usepackage{multirow}
\usepackage{appendix}
\usepackage{tikz}
\usepackage{authblk}
\usetikzlibrary{positioning,automata,decorations.markings}

\begin{document}

\title{\textbf {Controlo e Monitorização de Processos e Comunicação}\par\vspace{1cm}
\large Grupo Nº 60}
\author{Carlos Ferreira (A89509) \and José Alves (A89563) \and Luis Araújo (A86772)\vspace{2.5cm}}
\date{\today}

\begin{center}
    \begin{minipage}{0.9\linewidth}
        \centering
        \includegraphics[width=0.4\textwidth]{eng.jpeg}\par\vspace{1cm}
        \vspace{1.5cm}
        \href{https://www.uminho.pt/PT}
        {\color{black}{\scshape\LARGE Universidade do Minho}} \par
        \vspace{1cm}
        \href{https://www.di.uminho.pt/}
        {\color{black}{\scshape\Large Departamento de Informática}} \par
        \vspace{1.5cm}
        \maketitle
    \end{minipage}
\end{center}

\tableofcontents

\pagebreak

\chapter{Introdução}\label{chap:Introdução}
Foi nos proposto neste semestre a construção de um sistema de \textit{Controlo e Monitorização de Processos e Comunicação}. Para tal, teriamos de produzir uma comunicação entre um cliente e um servidor, armazenar informação sobre, nao só os comandos executados pelo cliente, como também e os seus \textit{outputs} e por fim controlar o tempo de execução de um processo.\\
Nos próximos capítulos deste relatório, iremos explicar com mais detalhe o problema proposto pelos docentes, as estratégias utilizadas para a resolução do mesmo, e por fim uma breve conclusão do projeto. 


\chapter{Problema proposto}\label{chap:Problema proposto}
Como já foi referido anteriormente, foi nos proposto a implementação de um sistema \textit{Controlo e Monitorização de Processos e Comunicação}. Para tal, foi nos indicado que teriamos de elaborar duas formas diferentes de comunicação com o utilizador, uma através da linha de comandos, e outro através de uma interface textual interpretada \textit{(shell)}.Por fim, tivemos ainda de desenvolver algumas funcionalidades:
\begin{itemize}
    \item\textbf {tempo-inactividade [-i]} tempo máximo de inactividade de comunicação num pipe anónimo
    \item\textbf {tempo-execucao [-m]} tempo máximo de execução de um tarefa
    \item\textbf {executar [-e]} executar uma tarefa
    \item\textbf {listar [-l]} listar as tarefas em execução
    \item\textbf {terminar [-t]} terminar uma tarefa em execução
    \item\textbf {historico [-r]} histórico de tarefas terminadas
    \item\textbf {ajuda [-h]} linhas de comandas da utilizaçao do sistema
    \item\textbf {output [-o]} \textit{standard outputs} produzido por uma tarefa já executada
\end{itemize}




\chapter{Resolução do problema}\label{chap:Resolução do problema}

\hypertarget{Client}{}
\section{Client}

\hypertarget{Server}{}
\section{Server}

\hypertarget{Parse}{}
\section{Passe}

\hypertarget{Output}{}
\section{Output}



\chapter{Conclusão}\label{chap:Conclusão}
Ao longo do desenvolvimento do trabalho, o grupo deparou-se com algumas decisões em termos de implementaçao, sendo uma delas a utilização de memória dinâmica.
Porem, face ao problema apresentado e analisando criticamente a solução proposta concluímos que cumprimos todos os requisitos propostos, criando uma arquitetura \textit{Cliente,Servidor}. 
Por fim, avaliamos este projeto como um sucesso perante o que foi sugerido pelos docentes desta unidade curricular.
\end{document}